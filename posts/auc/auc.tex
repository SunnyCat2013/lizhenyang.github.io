\documentclass{article}
\usepackage{ctex}
\usepackage{graphicx}
\usepackage{verbatim}
\usepackage{amsmath}
\usepackage{hyperref}
%\title{AUC 学习笔记}
\author{Li Zhenyang\thanks{lizhenyang\_2008@163.com}}
\date{November 2019}
\begin{document}
   \maketitle
   \section{举个例子}
    看图\ref{auc-and-distribution} 当设置一个阈值的时候,右边都被认为是 positive ,左边是 negtive 。那么 
    $$
    AUC = \int_{t=0}^{t=1} \frac{N_{l=1, s > t}}{N_{l=1}} d\frac{N_{l=0, s > t}}{N_{l=0}}
    $$

    详细的讲,阈值 $t$ 从零到壹,每一个阈值都会有一个对应的数值对($\frac{N_{l=1, s > t}}{N_{l=1}}, \frac{N_{l=0, s > t}}{N_{l=0}}$)。根据图示理解起来的话,就是(正样本在阈值右边的面积比上正样本全部的面积,负样本右边的面积比上负样本全部的面积)。
    无论正样本负样本,都是阈值右边的面积比上当前样本的全部面积。

\begin{comment}
   \section{AUC 是什么?}
   \section{计算代码}
   \section{一些思考}
   根据 [模型评估指标AUC(area under the curve)](https://blog.csdn.net/liweibin1994/article/details/79462554) 这个文章里面的一个图,AUC 可以被认为是正负样本分布的重合程度。重合程度越高,AUC 越大。当正负样本分布较接近时,AUC 接近 0.5。
\end{comment}
   \begin{figure}[h]
       \includegraphics[width=0.8\textwidth]{threshold-auc.png}
       \caption{auc-and-distribution}
       \label{auc-and-distribution}
   \end{figure}


   参考图 \ref{auc-and-distribution},AUC 其实就是竖线右边的部分占篮红区域的比例。

    \section{References} 
        \begin{itemize}
            \item https://tracholar.github.io/machine-learning/2018/01/26/auc.html 这里面解释概率的部分,写的不错。
             \item \url{https://www.dataschool.io/roc-curves-and-auc-explained/}
             \item \url{https://people.inf.elte.hu/kiss/13dwhdm/roc.pdf}
             \item \url{http://www.navan.name/roc/} 不同阈值下的 tpr/fpr 演示。
        \end{itemize}
\end{document}
