\documentclass{article}
\usepackage{ctex}
\usepackage{graphicx}
\usepackage{hyperref}
\title{AUC 学习笔记}
\author{Li Zhenyang\thanks{lizhenyang\_2008@163.com}}
\date{October 2019}
\begin{document}
   \maketitle
   \section{AUC 是什么?}
   \section{举个例子}
   \section{计算代码}
   \section{一些思考}
   根据 [模型评估指标AUC(area under the curve)](https://blog.csdn.net/liweibin1994/article/details/79462554) 这个文章里面的一个图,AUC 可以被认为是正负样本分布的重合程度。重合程度越高,AUC 越大。当正负样本分布较接近时,AUC 接近 0.5。
   \begin{figure}[h]
       \includegraphics[width=0.8\textwidth]{threshold-auc.png}
       \label{auc-and-distribution}
       \caption{auc-and-distribution}
   \end{figure}


   参考图 \ref{auc-and-distribution},AUC 其实就是竖线右边的部分占篮红区域的比例。

    \section{References} 
     https://tracholar.github.io/machine-learning/2018/01/26/auc.html 这里面解释概率的部分,写的不错。
     \url{https://www.dataschool.io/roc-curves-and-auc-explained/}
     \url{https://people.inf.elte.hu/kiss/13dwhdm/roc.pdf}
     \url{http://www.navan.name/roc/} 不同阈值下的 tpr/fpr 演示。
\end{document}
