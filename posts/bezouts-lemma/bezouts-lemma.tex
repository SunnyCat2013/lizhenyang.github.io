\documentclass{article}
\usepackage{ctex}
\usepackage{amsfonts}

\usepackage[all]{background}
\backgroundsetup{contents=Draft/CY\_TEC,color=blue,opacity=0.3,scale=10}
\usepackage{amsmath}
\newtheorem{bl}{Bezou's Lemma 定理}
\usepackage{hyperref}
\hypersetup{
    colorlinks=true,
    linkcolor=blue,
    filecolor=magenta,
    urlcolor=cyan,
}

\urlstyle{same}

\title{Bezout's Lemma 学习笔记}
%\author{Li Zhenyang \thanks{lizhenyang\_2008@163.com}}
\date{9, November 2019}
\begin{document}
   \maketitle
   \section{Bezout's Lemma 是什么?}
   \begin{bl}
       $\alpha, \beta \in \mathbb{Z}, \alpha, \beta \ne 0$。
       如果 $d_0 = gcd(\alpha, \beta)$
       $\alpha x + \beta y = d_0$ 有无穷多的解。且 $d_0 = min\{\alpha x + \beta y| \alpha, \beta \in \mathbb{Z}\}$。
       如果 $\alpha, \beta$ 互质($\alpha \bot \beta$),那么 $\alpha x + \beta y = 1$ 有解。
       
   \end{bl}
   
   \section{证明}
   如果 $\alpha \ \beta$ 中有一个为 0,比如 $\alpha = 0$,则 $gcd(0, \beta) = \beta$,那么 $ \alpha x + \beta y =  0 * x + \beta y = d_0 = \beta$。
   此时等式的解$(\forall x,  y = 1)$。

   当 $\alpha \ \beta$ 均不为 0,设 $A=\{\alpha x + \beta y;(x;y) \in \mathbb{Z}^2\}$。

   首先证明 $minA = gcd(\alpha, \beta)$。假设 $minA = d_0 = \alpha x_0 + \beta y_0 $。
   $A$ 中任意一个正元素 $p=\alpha x_1 + \beta y_1$。
   有 
   $$
   p = q * d_0 + r, q \in \mathbb{Z}^+, r \in [0, d_0)
   $$
   $$
   r = p - q * d_0 = \alpha x_1 + \beta y_1 - q * (\alpha x_0 + \beta y_0) = \alpha (x_1 - q * x_0) + \beta (y_1 - q * y_0) \in A
   $$
   因为 $r \in [0, d_0) \land r \in A$ ,即 $r$ 属于 $A$ 且不是 $A$ 的最小正整数,那么 $r = 0$。
   则 $d_0 | p$ ,即任意的  $p \in A$ 可以被 $minA = d_0$ 整除。
   可以推出 $d_0 | \alpha, d_0 | \beta$,因为 $\alpha, \beta \in A$。
   因此,$ d_0$ 是 $\alpha, \beta $ 的公因数。
   另一方面,对于 $\alpha, \beta $ 的任意公因数$d, \alpha = kd, \beta = ld$,
   $$
   d_0 = \alpha x_0 + \beta y_0 = (k x_0 + l y_0) d
   $$
   因此 $d | d_0$,所以,$d_0$ 是 $\alpha, \beta$ 的最大公因数。

   这时候,我们来构造一组方程的解 
   $$
   \{(x_0 + \frac{k\beta}{d}, y_0 - \frac{k\alpha}{d}) | k \in \mathbb{Z}\}
   $$
   知方程有无穷解。
   \section{References} 
   \url{https://artofproblemsolving.com/wiki/index.php/Bezout\%27s\_Lemma}
   \url{https://zh.wikipedia.org/wiki/%E8%B2%9D%E7%A5%96%E7%AD%89%E5%BC%8F}
\end{document}
